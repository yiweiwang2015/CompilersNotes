\ifx\PREAMBLE\undefined
\documentclass{report}
\usepackage[format = hang, font = bf]{caption}
\usepackage{graphicx}
\usepackage{array}
\usepackage{amsmath}
\usepackage{mathtools}
\usepackage{boxedminipage}
\usepackage{listings}
\usepackage{makecell}%diagonal line in table
\usepackage{float}%allowing forceful figure[H]
\usepackage{xcolor}
\usepackage{amsfonts}%allowing \mathbb{R}
\usepackage{alltt}
\usepackage{algorithmicx}
\usepackage[chapter]{algorithm} 
%chapter option ensures that algorithms are numbered within each chapter rather than in the whole article
\usepackage[noend]{algpseudocode} %If end if, end procdeure, etc is expected to appear, remove the noend option
\usepackage{xspace}
\usepackage{color}
\usepackage{url}
\def\UrlBreaks{\do\A\do\B\do\C\do\D\do\E\do\F\do\G\do\H\do\I\do\J\do\K\do\L\do\M\do\N\do\O\do\P\do\Q\do\R\do\S\do\T\do\U\do\V\do\W\do\X\do\Y\do\Z\do\[\do\\\do\]\do\^\do\_\do\`\do\a\do\b\do\c\do\d\do\e\do\f\do\g\do\h\do\i\do\j\do\k\do\l\do\m\do\n\do\o\do\p\do\q\do\r\do\s\do\t\do\u\do\v\do\w\do\x\do\y\do\z\do\0\do\1\do\2\do\3\do\4\do\5\do\6\do\7\do\8\do\9\do\.\do\@\do\\\do\/\do\!\do\_\do\|\do\;\do\>\do\]\do\)\do\,\do\?\do\'\do+\do\=\do\#\do\-}
\usepackage[breaklinks = true]{hyperref}
\lstset{language = c++, breaklines = true, tabsize = 2, numbers = left, extendedchars = false, basicstyle = {\ttfamily \footnotesize}, keywordstyle=\color{blue!70}, commentstyle=\color{red!70}, frame=shadowbox, rulesepcolor=\color{red!20!green!20!blue!20}, numberstyle={\color[RGB]{0,192,192}}}
\mathchardef\myhyphen="2D
% switch-case environment definitions
\algblock{switch}{endswitch} 
\algblock{case}{endcase}
%\algrenewtext{endswitch}{\textbf{end switch}} %If end switch is expected to appear, uncomment this line.
\algtext*{endswitch} % Make end switch disappear
\algtext*{endcase}
\begin{document}
\fi
\chapter{Introduction}
\section{Interpreter and compiler}
There are two approaches to implement a programing language: compilers and interpreters.

Interpreter is an online approach, i.e. the work done by the interpreter is part of running the problem. The program we write and the data on which we wish to run the program are inputted into the interpreter, after which the output is produced by the interpreter. 

Compiler is an offline approach, i.e. whatever the compiler does is the pre-processing of the program, and it does not take part in the actual execution of the program on the data. The program is translated into an executable by the compiler, and the data is passed to the executable, which then outputs the result.
\section{History}
In 1954, IBM developed the 704 machine. The customers found that the softwares cost more than the hardware, though the hardware already costs a lot. This inspired a lot of people to try to improve the productiveness of programing, among whom was John Backus. He developed ``Speedcoding'', which from today's the point of view is an interpreter. Speedcoding made it much faster to develop programs, but the programs developed with it run much slower and also occupies too much memory. Backus continued to develop the FORTRAN project, which is an abbr. for FORmula TRANslation. With FORTRAN I, he took a compiler approach: formulas written by programmers are translated into a form that could be understood by the machine. FORTRAN I is a successful project not only in the sense that it was soon adopted by most developers back in the 1950s, but also in the sense that its outline is still preserved by modern compilers. A compiler contains 5 phases:
\begin{description}
\item [Lexical analysis] Syntactic.
\item [Parsing] Syntactic.
\item [Semantic analysis] Types, scopes, etc.
\item [Optimization]
\item [Code generation] Translation into another language.
\end{description}
%\section{Structure of a compiler}
%\section{The economy of programming languages}
\ifx\PREAMBLE\undefined
\end{document}
\fi